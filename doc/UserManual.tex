\documentclass[11pt]{article}
%\usepackage[german]{babel}
\usepackage[utf8]{inputenc}
\inputencoding{utf8}
\usepackage[T1]{fontenc}
\usepackage{textcomp}
\usepackage{graphicx}
\usepackage{geometry}
\geometry{a4paper, top=1cm, body={16cm,24cm}}
\special{papersize=8.5in,11in}

\usepackage{listings}
\usepackage{xcolor}
\definecolor{backcolour}{rgb}{0.95,0.95,0.92}
\lstdefinestyle{mystyle}{
    backgroundcolor=\color{backcolour},   
    basicstyle=\ttfamily\footnotesize,
    breakatwhitespace=false,         
    breaklines=true,                 
    keepspaces=true,                 
    numbers=none,                    
    numbersep=5pt,                  
    showspaces=false,                
    showstringspaces=false,
    showtabs=false,                  
    tabsize=2
}
\lstset{style=mystyle}

\sloppy

\title{User Manual for TEUFEL \\ THz Emission from Undulators and Free-Electron Lasers}

\author{U. Lehnert\\
\footnotesize{Helmholtz-Center Dresden-Rossendorf, Institute of Radiation Physics}\\
\footnotesize{PF 510119, D-01314 Dresden}\\
\footnotesize{e-mail : U.Lehnert@hzdr.de}}

\date{\today\\[1cm]}

\begin{document}

\maketitle
\tableofcontents

\section{Physics model and code description}

\subsection{Parallelization}

\subsection{Installation}

\section{Input files}

The input to TEUFEL consists of a human readable XML file. The content should follow
the template given here. The code would complain if any of the essential sections
listed here were missing. Some example input files are contained in the \verb|examples/| folder.

\begin{lstlisting}
<?xml version="1.0"?>
<teufel description="description of the simulation case"
    author="author name and date of creation">
    <lattice />
    <beam />
    <tracking method="Vay" delta_t="0.001/_c" n="4000" />
    <observer />
</teufel>
\end{lstlisting}

\subsection{Lattice description}

The lattice node describes all external fields acting on the beam.

\subsubsection{Background fields}
Homogeneous, time-constant background fields can be defined as either electrical or magnetic
fields or both.
\begin{lstlisting}
<background>
    <E x="..." y="..." z="..." />
    <B x="..." y="..." z="..." />
</background>
\end{lstlisting}

%undulator
%    planar
%    transverse gradient
%wave
%    gaussian
%dipole
%    hard edge
%background

\subsection{Beam description}

%particle
%bunch

\subsection{Tracking definitions}

%method
%watch

\subsection{Definition of Observations}

%snapshot
%screen
%mesh
%point

\subsection{The calculator}

A simple calculator and variable engine is integrated with
the input parser. All numerical input values can be computed terms.
Variable definitions can appear at any position within the file.
All variable definitions are parsed first within a section of the input file
regardless of their sequence with respect to other entries.
However, definitions made within subsections become visible in
the superior section only after the corresponding subsection has been parsed.

A variable definition is made with the entry
\begin{lstlisting}
<calc var="name" eq="value" print="entity [unit] = " />
\end{lstlisting}
The value itself may be computed from already defined variables.
At the time of the definition a printout of the given text along with the
assigned value is generated. The print attribute can be omitted if no printout is desired.
A few variables are pre-defined. All of those have names starting with an underscore.
\\[1ex]
\begin{tabular}{lll}
\_c & 2.99792458e8 & speed of light [m/s] \\
\_e & 1.6021766208e-19 & elementary charge [As] \\
\_mec2 & 0.5109989461e6 & electron rest mass [eV] \\
\_eps0 & 8.854187817e-12 & vacuum permittivity [As/Vm] \\
\_pi & $\pi$ & $\pi$ \\
\end{tabular}

\section{Scripts for creating input files}

\section{Scripts for output visualization}

\end{document}


