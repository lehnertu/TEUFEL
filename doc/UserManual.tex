\documentclass[11pt]{article}
%\usepackage[german]{babel}
\usepackage[utf8]{inputenc}
\inputencoding{utf8}
\usepackage[T1]{fontenc}
\usepackage{textcomp}
\usepackage{graphicx}
\usepackage{geometry}
\geometry{a4paper, top=1cm, body={16cm,24cm}}
\special{papersize=8.5in,11in}

\usepackage{titlesec}
\titlespacing*{\section}
{0pt}{10.0ex plus 1ex minus .5ex}{2.0ex plus .5ex}
\titlespacing*{\subsection}
{0pt}{2.5ex plus 1ex minus .5ex}{2.0ex plus .5ex}
\titlespacing*{\subsubsection}
{0pt}{2.5ex plus 1ex minus .5ex}{1.0ex plus .5ex}

\usepackage{listings}
\usepackage{xcolor}
\definecolor{backcolour}{rgb}{0.95,0.95,0.92}
\lstdefinestyle{mystyle}{
    backgroundcolor=\color{backcolour},   
    basicstyle=\ttfamily\footnotesize,
    breakatwhitespace=false,         
    breaklines=true,                 
    keepspaces=true,                 
    numbers=none,                    
    numbersep=5pt,                  
    showspaces=false,                
    showstringspaces=false,
    showtabs=false,                  
    tabsize=2
}
\lstset{style=mystyle}

\sloppy

\title{User Manual for TEUFEL \\ THz Emission from Undulators and Free-Electron Lasers}

\author{U. Lehnert\\
\footnotesize{Helmholtz-Center Dresden-Rossendorf, Institute of Radiation Physics}\\
\footnotesize{PF 510119, D-01314 Dresden}\\
\footnotesize{e-mail : U.Lehnert@hzdr.de}}

\date{\today\\[1cm]}

\begin{document}

\maketitle
\tableofcontents

\section{Physics model and code description}

\subsection{Parallelization}

TEUFEL is designed to make use of clusters of multi-core nodes.
The particles are distributed over the nodes (each tracking its own sub-set of particles)
while the particle coordinates are communicated between the nodes
after every tracking step using the message passing interface (OpenMPI).

After the tracking every node computes the observed fields from its own set of particles.
This is the most time-consuming step of the computation. It is addidionally parallelized
using the shared-memory model Open-MP. This way all CPU's available on one node
can cooperate to compute the fields. Only a single copy of the fields (which can be very large)
needs to be held in memory. After every node has computed the fields from its own particles
the total fields are gethered onto the master node (using MPI) and written to file.

As an example, when running a problem on an 8-core workstation where the fields
(and necessary communication buffers) fit into the memory twice,
one would start TEUFEL with 2 nodes each using 4 CPU's
\begin{lstlisting}
mpirun -n 2 --cpus-per-rank 4 --oversubscribe ./build/teufel input.xml
\end{lstlisting}

On a cluster one would have to submit a batch job with the according information.

\subsection{Installation}

\subsection{Testing}

\section{Input files}

The input to TEUFEL consists of a human readable XML file. The content should follow
the template given here. The code would complain if any of the essential sections
listed here were missing. Some example input files are contained in the \verb|examples/| folder.

\begin{lstlisting}
<?xml version="1.0"?>
<teufel description="description of the simulation case"
    author="author name and date of creation">
    <lattice />
    <beam />
    <tracking method="Vay" delta_t="0.001/_c" n="4000" />
    <observer />
</teufel>
\end{lstlisting}

\subsection{Lattice description}

The lattice node describes all external fields acting on the beam.

\subsubsection{Background fields}
Homogeneous, time-constant background fields can be defined as either electrical or magnetic
fields or both.
\begin{lstlisting}
<background>
    <E x="..." y="..." z="..." />
    <B x="..." y="..." z="..." />
</background>
\end{lstlisting}

%undulator
%    planar
%    transverse gradient
%wave
%    gaussian
%dipole
%    hard edge
%background

\subsubsection{Wave-packet}
This describes a gaussian pulse of a gaussian beam of electromagnetic radiation.
The radiation wavelength, complex electric field amplitude and Rayleigh length of
the optical waist must be given in the \verb|<field>| node. The \verb|<position>| node
gives the position of the optical waist. The peak of the wave packet passes through this waist
at the time given in the \verb|<position>| node. The wavepacket propagates
in positive $z$ direction by default. The example given here also shows
one way to calculate the field parameters.
\begin{lstlisting}
<wave name="seed" type="packet">
	<!-- compute the seed field -->
	<calc print="seed power [W] = " var="P0" eq="1e9"/>
	<calc print="Rayleigh range [m] = " var="zR" eq="1.0"/>
	<calc print="waist radius w0 [m] = " var="w0" eq="sqrt(lambda_S*zR/_pi)"/>
	<calc print="peak intensity I0 [W/m²] = " var="I0" eq="2.0*P0/(_pi*w0*w0)"/>
	<calc print="peak field E0 [V/m] = " var="E0" eq="sqrt(2*I0/_eps0/_c)"/>
	<position x="0.0" y="0.0" z="0.0" t="1.0/_c"/>
	<field lambda="10.0e-6" ReA="E0" ImA="0.0" rayleigh="zR" tau="50.0e-6/_c"/>
</wave>
\end{lstlisting}

\subsection{Beam description}

\subsubsection{Coordinate systems}

TEUFEL is per default using a right-handed cartesian laboratory system which could be described
as ($x$--"north", $y$--"up", $z$--"east") but any other definitions can be used as long as it is
a right-handed cartesian system. All lattice fields can be placed freely in the chosen coordinate system.

One should be aware that some post-processing routines interpret the $x$ and $y$ components
of the recorded fields to be "right" and "up" with respect to the propagation direction
of the beam. If one wants this to hold strictly one has to start the beam in the negative $z$ direction.

\subsubsection{SDDS files}

\begin{lstlisting}
<beam>
    <SDDS file="tests/test.sdds">
        <dir x="..." y="..." z="..." />
    </SDDS>
</beam>
\end{lstlisting}

Beams to be tracked using TEUFEL can be imported from other codes (e.g. ELEGANT)
by reading SDDS files. SDDS files are imported as a <bunch>, several bunches can be
be combined into on beam to be tracked. TEUFEL expects two parameters to be present
in the header of the SDDS file - "Particles" giving the total number and
"Charge" holding the total charge of all particles containd in the file.
Particle coordinates are read from 6 columns in the SDDS file:
\\[1ex]
\begin{tabular}{lll}
x & [m] & horizontal displacement (right) from the beam axis \\
xp & [ ] & horizontal angle with respect to the beam axis (${\rm d}x/{\rm d}z$) \\
y & [m] & vertical displacement (up) from the beam axis \\
yp & [ ] & vertical angle with respect to the beam axis (${\rm d}y/{\rm d}z$) \\
p & [ ] & total particle momentum $\beta\gamma = p/m_e c$ \\
t & [s] & arrival time of the particle at the observation plane \\
\end{tabular}\\

The beam will be setup propagating in the direction $s$ given in the import statement.
The x values read from the file will be mapped to a vector in the $x$-$z$ plane
perpendicular to $s$. The y values will be mapped such that $x$-$y$-$s$ forms
a right-handed cartesian system. If $s$ is not given it will be chosen as the positive $z$
direction in which case x is mapped to $x$ and y is mapped to $-y$.

The resulting bunch will be created with all particles starting at $t=0$.
All particles are shifted along their individual propagation directions
to account for the difference to the average arrival time of the particles in the file.

\subsubsection{particle}

\subsubsection{bunch}

\subsection{Tracking definitions}

The number of steps must be an integer value (cannot be calculated)!
\begin{lstlisting}
<tracking method="Vay" delta_t="dt" n="1000000" />
\end{lstlisting}

%method
%watch

\subsection{Definition of Observations}

%snapshot
%screen
%mesh
%point

\subsection{The calculator}

A simple calculator and variable engine is integrated with
the input parser. All numerical input values can be computed terms.
Variable definitions can appear at any position within the file.
All variable definitions are parsed first within a section of the input file
regardless of their sequence with respect to other entries.
However, definitions made within subsections become visible in
the superior section only after the corresponding subsection has been parsed.

A variable definition is made with the entry
\begin{lstlisting}
<calc var="name" eq="value" print="entity [unit] = " />
\end{lstlisting}
The value itself may be computed from already defined variables.
At the time of the definition a printout of the given text along with the
assigned value is generated. The print attribute can be omitted if no printout is desired.
A few variables are pre-defined. All of those have names starting with an underscore.
\\[1ex]
\begin{tabular}{lll}
\_c & 2.99792458e8 & speed of light [m/s] \\
\_e & 1.6021766208e-19 & elementary charge [As] \\
\_mec2 & 0.5109989461e6 & electron rest mass [eV] \\
\_eps0 & 8.854187817e-12 & vacuum permittivity [As/Vm] \\
\_pi & $\pi$ & $\pi$ \\
\end{tabular}

\section{Scripts for creating input files}

\section{Scripts for output visualization}

\section{Examples}

\subsection{ELBE U300}

Notice that the screen has been shifted by half a pixel size
such that the (Nx/2, Ny/2) pixel is exactly on the axis.
That is the arrangement expected when importing the screen fields into pyOPC
for further propagation of single-frequency optical beams.

\end{document}


