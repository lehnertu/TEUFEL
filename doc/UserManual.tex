\documentclass[11pt]{article}
%\usepackage[german]{babel}
\usepackage[utf8]{inputenc}
\inputencoding{utf8}
\usepackage[T1]{fontenc}
\usepackage{graphicx}
\usepackage{geometry}
\geometry{a4paper, top=1cm, body={16cm,24cm}}
\special{papersize=8.5in,11in}

\usepackage{listings}
\usepackage{xcolor}
\definecolor{backcolour}{rgb}{0.95,0.95,0.92}
\lstdefinestyle{mystyle}{
    backgroundcolor=\color{backcolour},   
    basicstyle=\ttfamily\footnotesize,
    breakatwhitespace=false,         
    breaklines=true,                 
    keepspaces=true,                 
    numbers=none,                    
    numbersep=5pt,                  
    showspaces=false,                
    showstringspaces=false,
    showtabs=false,                  
    tabsize=2
}
\lstset{style=mystyle}

\sloppy

\title{User Manual for TEUFEL - THz Emission from Undulators and Free-Electron Lasers}

\author{U. Lehnert\\
\footnotesize{Helmholtz-Center Dresden-Rossendorf}
\footnotesize{Institute of Radiation Physics}\\
\footnotesize{PF 510119, D-01314 Dresden}\\
\footnotesize{e-mail : U.Lehnert@fzd.de}}

\date{\today\\[1cm]}

\begin{document}

\maketitle
\tableofcontents

\section{Physics model and code description}

\subsection{Parallelization}

\subsection{Installation}

\section{Input files}

The input to TEUFEL consists of a human readable XML file. The content should follow
the template given here. The code would complain if any of the essential sections
listed here were missing. Some example input files are contained in the \verb|examples/| folder.

\begin{lstlisting}
<?xml version="1.0"?>
<teufel description="description of the simulation case"
    author="author name and date of creation">
    <lattice />
    <beam />
    <tracking method="Vay" delta_t="0.001/_c" n="4000" />
    <observer />
</teufel>
\end{lstlisting}

\subsection{Lattice description}

%undulator
%    planar
%    transverse gradient
%wave
%    gaussian
%dipole
%    hard edge
%background

\subsection{Beam description}

%particle
%bunch

\subsection{Tracking definitions}

%method
%watch

\subsection{Definition of Observations}

%snapshot
%screen
%mesh
%point

\subsection{The calculator}

\section{Scripts for creating input files}

\section{Scripts for output visualization}

\end{document}


